\documentclass{myreport}

\usepackage{tabularray}
\usepackage{float}
\usepackage{graphicx}
\usepackage{codehigh}
\usepackage[normalem]{ulem}

% References
\bibliographystyle{copernicus.bst}

\begin{document}
\pagestyle{headings}

% Change figure (table, section) numbering (e.g., from 'Figure 1' to 'Figure S1')
\renewcommand{\thefigure}{S\arabic{figure}}
\renewcommand{\thetable}{S\arabic{table}}
\renewcommand{\thesection}{S\arabic{section}}
\renewcommand{\theequation}{S\arabic{equation}}

% Document must include
% ---------------------
%

%% Title
\title{SUPPLEMENTARY INFORMATION:\\
Global forest thickening}
\author{Marqués et al.}

\maketitle

\tableofcontents

\section{Data}

\begin{table}
  % \label{tab:datasets}
\centering
\fontsize{8}{10}\selectfont
\begin{tabular}{llp{4cm}p{4cm}l}
  \toprule
Dataset & N & Description & Filter & Reference \\
    \midrule
  nfi\_spain & 27642 & Spanish National Forest Inventory & No management intervention observed during monitoring & Paloma Ruiz Benito paloma.ruizb@uah.es Veronica Cruz Alonso veronica.cral@gmail.com \\
  nfi\_norway & 25156 & Norwegian National Forest Inventory & No management intervention observed during monitoring & Johannes Breidenbach johannes.breidenbach@nibio.no Oliver Moen Snoksrud oliver.snoksrud@nibio.no \\
  nfi\_sweeden & 15954 & Swedish National Forest Inventory & No management intervention observed during monitoring & Julian Tijerin-Triviño tijerin25@hotmail.com \\
  bnp & 9423 & Berchtesgaden National Park & Forest reserves & Michael Maroschek michael.maroschek@npv-bgd.bayern.de Rupert Seidl rupert.seidl@tum.de \\
  fia\_us & 7022 & Forest Inventory and Analysis, US & Forest reserves & Doser JW, Stanke H, Finley AO (2025). rFIA: Estimation of Forest Variables using the FIA Database. R package version 1.1.0, https://CRAN.R-project.org/package=rFIA \\
  aus\_plots & 6259 & Sustainable Timber Tasmania, Forestry Corporation of NSW, Queensland, Victoria and Australia's Terrestrial Ecosystem Research Network & No management intervention observed during monitoring & David Forrester david.forrester@csiro.au Thomas Baker thomas.baker@sttas.com.au Shaun Suitor shaun.suitor@sttas.com.au Mike Sutton mike.sutton@fcnsw.com.au Michael Ngugi michael.ngugi@des.qld.gov.au John Neldner john.neldner@des.qld.gov.au Andrew Clark andrew.j.clark@delwp.vic.gov.au www.tern.org.au \\
  luquillo & 1993 & Luquillo & No management intervention observed during monitoring & Jess Zimmerman jesskz@ites.upr.edu \\
  nfi\_switzerland & 1972 & Swiss National Forest Inventory & No management intervention observed during the last 70 years & Brigitte Rohner brigitte.rohner@wsl.ch \\
  scbi & 1572 & Smithsonian Conservation Biology Institute & No management intervention observed during monitoring & Kristina J. Anderson-Teixeira TeixeiraK@si.edu  William J. McShea mcsheaw@si.edu Norman A. Bourg bourgn@si.edu \\
  wuls & 1416 & Białowieża National Park & Forest reserves & Bogdan Brzeziecki bogdan_brzeziecki@sggw.edu.pl \\
  wytham & 1200 & Wytham Woods & No management intervention observed during monitoring & Yadvinder Malhi \\
  serc & 1026 & Smithsonian Environmental Research Center & No management intervention observed during monitoring & Sean McMahon mcmahons@si.edu \\
  pasoh & 1007 & Pasoh & No management intervention observed during monitoring & Yao Tze Leong yaotzeleong@frim.gov.my \\
  df\_rainfor & 988 & Amazon Forest Inventory Network (RAINFOR) & No management intervention ocurred & Esquivel-Muelbert, A., Banbury Morgan, R., Brienen, R. et al. Increasing tree size across Amazonia. Nat. Plants 11, 2016–2025 (2025). https://doi.org/10.1038/s41477-025-02097-4 \\
  nfr\_swi & 729 & Swiss Natural Forest Reserves & Forest reserves & Martina Hobi martina.hobi@wsl.ch Harald Bugmann harald.bugmann@env.ethz.ch \\
  forst & 537 & Forest Research Institute Baden-Württemberg & Forest reserves & \\
  palanam & 484 & Palanam & Perry S. Ong ongperry@yahoo.com & \\
  unito & 311 & University of Turin  & Forest reserves & Renzo Motta renzo.motta@unito.it \\
  uholka & 200 & Uholka-Shyrokyi Luh & Forest reserves & Jonas Stillhard jonas.stillhard@wsl.ch \\
  df\_forestplots & 149 & Forest Inventory Network & No management intervention ocurred & Yadvinder Malhi yadvinder.malhi@ouce.ox.ac.uk Huanyuan Zhang huanyuan.zhang@ouce.ox.ac.uk \\
  mudumalai & 126 & Mudumalai & No management intervention observed during monitoring & Sukumar Raman rsuku@iisc.ac.in HS Suresh sureshhs@iisc.ac.in \\
  lwf\_tree & 114 & Bavarian Institute of Forestry & Forest reserves & Markus Blaschke markus.blaschke@lwf.bayern.de \\
  nwfva\_tree & 84 &  Northwest German Forest Research Institute (NW-FVA) & Forest reserves & Peter Meyer peter.meyer@nw-fva.de \\
  incds &  75 & National Institute for Research-Development in Forestry ‘‘Marin Drăcea’’ Department of Forest & Forest reserves & Any Mary Petritan apetritan@gmail.com Cătălin Petritan petritan@unitbv.ro\\
  tuzvo\_tree &  63 & Technical University in Zvolen & Forest reserves & Stanislav Kucbel kucbel@tuzvo.sk Peter Jalovia jaloviar@tuzvo.sk\\
  iberbas &  57 & Institute of Biodiversity and Ecosystem Research, Bulgarian Academy of Sciences & Forest reserves & Tzvetan Zlatanov tmzlatanov@gmail.com \\
  efm\_swi &  51 & Experiemental Forest Management plots & No management intervention observed during monitoring & Jonas Glatthorn jonas.glatthorn@wsl.ch \\
  france & 47 & French plots & No management intervention observed during monitoring & Georges Kunstler georges.kunstler@inrae.fr \\
  greece\_stand &  40 & Greek plots & No management intervention observed during monitoring & Gavriil Spyroglou spyroglou@fri.gr Nikos Fillas nfyllas@aegean.gr \\
  czu &  24 & Czech University of Life Sciences Prague & Forest reserves & Miroslav Svoboda svobodam@fld.czu.cz \\
  ul\_tree &  23 & University of Ljubljana, Slovenia & Forest reserves & Thomas Nagel tom.nagel@bf.uni-lj.si \\
  urk &  12 & Roztocze National Park, Poland & Forest reserves & Srdjan Keren srdjan.keren@urk.edu.pl Zbigniew Maciejewski \\
  nbw & 7 & NPV-BW & Forest reserves & Marco Heurich marco.heurich@npv-bw.bayern.de Isabelle Klein isabelle.klein@npv-bw.bayern.de \\
  \bottomrule
  \end{tabular}
\caption{Constituent forest dataset sizes and descriptions.}
\end{table}

\begin{figure}
\centering
\includegraphics[width=0.9\textwidth]{../figures/fig_hist_year.pdf}
\caption{Distribution of forest census data over time, grouped by biome (a-f). Dataset names are explained in Tab. xxxx.}
\end{figure}


\begin{figure}
\centering
\includegraphics[width=0.9\textwidth]{../figures/distribution_length.pdf}
\caption{Distribution of the total length of the time series per forest plot, separated by biomes. The total length corresponds to the difference in the observation year of the first and last available forest inventory for each plot.}
\end{figure}


\begin{figure}
\centering
\includegraphics[width=\textwidth]{../figures/gg_sitepoints.pdf}
\caption{Distribution of forest plots (red circles) and forest biomes.}
\end{figure}

\clearpage

\section{Self-thinning trends}

\begin{figure}
\centering
\includegraphics[width=\textwidth]{../figures/stl_longplots.pdf}
\caption{Self-thinning relation across biomes with example long-term forest monitoring plots highlighted.}
\end{figure}

\begin{figure}
\centering
\includegraphics[width=\textwidth]{../figures/slope_distribution.pdf}
\caption{Self-thinning relation across biomes with example long-term forest monitoring plots highlighted.}
\end{figure}

% XXX todo: visualise evolution of individual forest plots for a small subset of plots, selected based on length of available time series.

% Example for how to include a table, generated externally
% Uses library xtable.
% Function create_table_latex() is included also in this repo.
% R code that was used to create this example table:
% df_experiments_co2_asat <- df5 |>
%   filter(myvar == "asat") |>
%   select(exp, myvar) |>
%   left_join(
%     df |>
%       select(exp, myvar = response, citation),
%     by = join_by(exp, myvar)
%   ) |>
%   select(exp, citation) |>
%   distinct() |>
%
%   # format for processing with latex
%   mutate(citation = paste0('\\', "cite{", citation, "}")) |>
%   group_by(exp) |>
%   summarise(
%     citation = paste0(unique(citation), collapse = ", ")
%   ) |>
%   rename(
%     Experiment = exp,
%     Reference = citation
%   )

% % latex table generated in R 4.5.0 by xtable 1.8-4 package
% Thu Nov 27 09:47:09 2025
\begin{table}[ht]
\centering
\begin{tabular}{lll}
  \hline
Biome & Mean & SD \\ 
  \hline
Boreal Forests/Taiga & 0.30 & 0.06 \\ 
  Mediterranean Forests & 2.35 & 0.06 \\ 
  Temperate Broadleaf \& Mixed Forests & 0.91 & 0.03 \\ 
  Temperate Conifer Forests & 1.18 & 0.06 \\ 
  Tropical \& Subtropical Moist Broadleaf Forests & 0.16 & 0.07 \\ 
  Tropical Dry Broadleaf Forests & -0.38 & 0.46 \\ 
  \end{tabular}
\caption{Percentage change of forest stand density.} 
\end{table}


\begin{table}
\centering
\begin{tabular}{lll}
  \toprule
Biome & Mean & SE \\
  \midrule
  Boreal Forests/Taiga & 0.30 & 0.06 \\
  Mediterranean Forests & 2.35 & 0.06 \\
  Temperate Broadleaf \& Mixed Forests & 0.91 & 0.03 \\
  Temperate Conifer Forests & 1.18 & 0.06 \\
  Tropical \& Subtropical Moist Broadleaf Forests & 0.16 & 0.07 \\
  Tropical Dry Broadleaf Forests & -0.38 & 0.46 \\
  \bottomrule
  \end{tabular}
\caption{Mean estimate and standard error (SE) of percentage change (\%/yr) of forest stand density (number of trees per ha) by biome, determined from quantile regressions on bootstrapped data samples.}
\end{table}


\begin{figure}
\centering
\includegraphics[width=\textwidth]{../figures/hist_percent_change.pdf}
\caption{Distribution of percentage change (\%/yr) in stand density (number of trees per ha) by biome.}
\end{figure}


\begin{figure}
\centering
\includegraphics[width=\textwidth]{../figures/fdisturbed.pdf}
\caption{Trends in the fraction of disturbed forest plots, by biome. Fraction values are logit-transformed. The corresponding un-transformed values are indicated by the right y-axis in each plot. No regression fit is shown for tropical dry broadleaf forests (b) as only two points are available with non-zero values for the disturbed fraction.}
\end{figure}

\clearpage

\section{Environmental drivers}

\begin{table}
\centering
\scriptsize
\begin{talltblr}[         %% tabularray outer open
caption={Regression Results},
]                     %% tabularray outer close
{                     %% tabularray inner open
colspec={Q[]Q[]Q[]Q[]Q[]},
column{2,3,4,5}={}{halign=c,},
column{1}={}{halign=l,},
hline{31}={1,2,3,4,5}{solid, black, 0.05em},
}                     %% tabularray inner close
\toprule
& Complete & No PBR & No PBR, ORGC & No PBR, C:N \\ \midrule %% TinyTableHeader
scale(logQMD) & -0.861*** & -0.862*** & -0.862*** & -0.864*** \\
& [-0.865, -0.856] & [-0.867, -0.858] & [-0.867, -0.857] & [-0.869, -0.859] \\
scale(year) & 0.129*** & 0.130*** & 0.130*** & 0.132*** \\
& [0.126, 0.132] & [0.127, 0.133] & [0.128, 0.133] & [0.129, 0.135] \\
scale(tavg) & -0.033* & -0.026+ & -0.007 & -0.018 \\
& [-0.062, -0.003] & [-0.055, 0.002] & [-0.034, 0.020] & [-0.046, 0.011] \\
scale(ai) & 0.086*** & 0.095*** & 0.097*** & 0.087*** \\
& [0.066, 0.105] & [0.077, 0.114] & [0.079, 0.115] & [0.070, 0.105] \\
scale(ndep) & 0.153*** & 0.140*** & 0.146*** & 0.131*** \\
& [0.133, 0.174] & [0.120, 0.159] & [0.127, 0.166] & [0.112, 0.151] \\
scale(ORGC) & -0.039** & -0.048*** &  & -0.001 \\
& [-0.064, -0.014] & [-0.073, -0.024] &  & [-0.019, 0.017] \\
scale(PBR) & 0.004 &  &  &  \\
& [-0.012, 0.021] &  &  &  \\
scale(CNrt) & 0.057*** & 0.060*** & 0.031*** &  \\
& [0.035, 0.079] & [0.039, 0.081] & [0.015, 0.047] &  \\
scale(year) × scale(tavg) & 0.006** & 0.009*** & 0.013*** & 0.006** \\
& [0.002, 0.011] & [0.005, 0.013] & [0.009, 0.017] & [0.002, 0.010] \\
scale(year) × scale(ai) & -0.022*** & -0.018*** & -0.018*** & -0.017*** \\
& [-0.025, -0.019] & [-0.021, -0.015] & [-0.020, -0.015] & [-0.019, -0.014] \\
scale(year) × scale(ndep) & -0.016*** & -0.015*** & -0.015*** & -0.011*** \\
& [-0.019, -0.013] & [-0.018, -0.012] & [-0.018, -0.012] & [-0.013, -0.008] \\
scale(year) × scale(ORGC) & -0.012*** & -0.011*** &  & -0.028*** \\
& [-0.017, -0.008] & [-0.015, -0.007] &  & [-0.032, -0.025] \\
scale(year) × scale(PBR) & 0.006*** &  &  &  \\
& [0.002, 0.009] &  &  &  \\
scale(year) × scale(CNrt) & -0.021*** & -0.023*** & -0.028*** &  \\
& [-0.025, -0.017] & [-0.026, -0.019] & [-0.031, -0.025] &  \\
SD (Observations) & 0.176 & 0.178 & 0.178 & 0.178 \\
Num.Obs. & 36133 & 37652 & 37652 & 37652 \\
R2 Marg. & 0.521 & 0.530 & 0.531 & 0.527 \\
R2 Cond. & 0.980 & 0.980 & 0.980 & 0.980 \\
AIC & 17693.1 & 19142.8 & 19162.9 & 19315.9 \\
BIC & 17846.0 & 19279.3 & 19282.4 & 19435.4 \\
ICC & 1.0 & 1.0 & 1.0 & 1.0 \\
RMSE & 0.15 & 0.15 & 0.15 & 0.15 \\
\bottomrule
\end{talltblr}
\end{table}


% \begin{table}
\centering
\begin{talltblr}[         %% tabularray outer open
caption={Regression Results},
]                     %% tabularray outer close
{                     %% tabularray inner open
colspec={Q[]Q[]Q[]Q[]Q[]},
column{2,3,4,5}={}{halign=c,},
column{1}={}{halign=l,},
hline{31}={1,2,3,4,5}{solid, black, 0.05em},
}                     %% tabularray inner close
\toprule
& Complete & No PBR & No PBR, ORGC & No PBR, C:N \\ \midrule %% TinyTableHeader
scale(logQMD) & -0.646*** & -0.645*** & -0.645*** & -0.645*** \\
& (0.002) & (0.002) & (0.002) & (0.002) \\
scale(year) & 0.064*** & 0.064*** & 0.065*** & 0.065*** \\
& (0.001) & (0.001) & (0.001) & (0.001) \\
scale(tavg) & 0.025*** & 0.020** & 0.018* & 0.022** \\
& (0.007) & (0.007) & (0.007) & (0.007) \\
scale(ai) & 0.060*** & 0.059*** & 0.059*** & 0.059*** \\
& (0.004) & (0.004) & (0.004) & (0.004) \\
scale(ndep) & 0.042*** & 0.043*** & 0.043*** & 0.042*** \\
& (0.004) & (0.004) & (0.004) & (0.004) \\
scale(ORGC) & 0.007 & 0.005 &  & 0.006 \\
& (0.005) & (0.005) &  & (0.004) \\
scale(PBR) & 0.004 &  &  &  \\
& (0.003) &  &  &  \\
scale(CNrt) & -0.001 & -0.001 & 0.002 &  \\
& (0.004) & (0.004) & (0.003) &  \\
scale(year) × scale(tavg) & -0.007*** & -0.007*** & -0.004* & -0.007*** \\
& (0.002) & (0.001) & (0.001) & (0.001) \\
scale(year) × scale(ai) & -0.009*** & -0.008*** & -0.008*** & -0.007*** \\
& (0.001) & (0.001) & (0.001) & (0.001) \\
scale(year) × scale(ndep) & -0.005*** & -0.006*** & -0.006*** & -0.002 \\
& (0.001) & (0.001) & (0.001) & (0.001) \\
scale(year) × scale(ORGC) & -0.012*** & -0.011*** &  & -0.026*** \\
& (0.002) & (0.002) &  & (0.001) \\
scale(year) × scale(PBR) & 0.003+ &  &  &  \\
& (0.001) &  &  &  \\
scale(year) × scale(CNrt) & -0.019*** & -0.021*** & -0.027*** &  \\
& (0.002) & (0.001) & (0.001) &  \\
SD (Observations) & 0.109 & 0.110 & 0.110 & 0.111 \\
Num.Obs. & 22566 & 23570 & 23570 & 23570 \\
R2 Marg. & 0.728 & 0.712 & 0.710 & 0.711 \\
R2 Cond. & 0.979 & 0.980 & 0.979 & 0.979 \\
AIC & -13634.4 & -13910.0 & -13889.1 & -13732.0 \\
BIC & -13490.0 & -13780.9 & -13776.1 & -13619.0 \\
ICC & 0.9 & 0.9 & 0.9 & 0.9 \\
RMSE & 0.09 & 0.09 & 0.09 & 0.09 \\
\bottomrule
\end{talltblr}
\end{table}


\clearpage

% \begin{figure}
% \centering
% \includegraphics[width=0.8\textwidth]{../figures/coefs_env_quantilefilter.pdf}
% \caption{Coefficients of fixed effects of environmental factors on the self-thinning relationship. Data was filtered to retain observations from plots subject to dominant self-thinning dynamics based on quantiles within QMD bins, following Marqués et al. xxx}
% \end{figure}

\begin{figure}
\centering
\includegraphics[width=0.8\textwidth]{../figures/coefs_env_slopefilter.pdf}
\caption{Coefficients of fixed effects of environmental factors on the self-thinning relationship. Data was filtered to retain observations from plots subject to dominant self-thinning dynamics based on slopes.}
\end{figure}

\clearpage

\section{Global C sink}

\begin{figure}[ht!]
\centering
\includegraphics[width=\textwidth]{../figures/gg_map_sink_perforestarea_combined.pdf}
\caption{(a) C sink in aboveground biomass due to temporal changes in the self-thinning relationship. (b) Standard deviation of estimates across bootstraps. Values are expressed per unit forest area (gC m$^{-2}$ yr$^{-1}$).}
\end{figure}



%%%%%%%%%%%%%%%%%%%%%%%%%%%%%%%%%%%%%%%%%%%%%%%%%%%%%%%%%%%%%%%%%%%%%%%%%%%
\clearpage
\bibliography{references_global_forest_thickening.bib}


\end{document}

