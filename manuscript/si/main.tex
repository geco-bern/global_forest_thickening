\documentclass{myreport}

\usepackage{tabularray}
\usepackage{float}
\usepackage{graphicx}
\usepackage{codehigh}
\usepackage[normalem]{ulem}

% References
\bibliographystyle{copernicus.bst}

\begin{document}
\pagestyle{headings}

% Change figure (table, section) numbering (e.g., from 'Figure 1' to 'Figure S1')
\renewcommand{\thefigure}{S\arabic{figure}}
\renewcommand{\thetable}{S\arabic{table}}
\renewcommand{\thesection}{S\arabic{section}}
\renewcommand{\theequation}{S\arabic{equation}}

% Document must include
% ---------------------
%

%% Title
\title{SUPPLEMENTARY INFORMATION:\\
Global forest thickening}
\author{Marqués et al.}

\maketitle

\tableofcontents

\section{Data}

\begin{table}
  % \label{tab:datasets}
\centering
\begin{tabular}{lllll}
  \toprule
Dataset & N & Description & Filter & Reference \\
  \midrule
  nfi\_spain & 27642 &  &  & \\
  nfi\_norway & 25156 &  &  & \\
  nfi\_sweeden & 15954 &  &  & \\
  bnp & 9423 &  &  & \\
  fia\_us & 7022 &  &  & \\
  aus\_plots & 6259 &  &  & \\
  luquillo & 1993 &  &  & \\
  nfi\_switzerland & 1972 &  &  & \\
  scbi & 1572 &  &  & \\
  wuls & 1416 &  &  & \\
  wytham & 1200 &  &  & \\
  serc & 1026 &  &  & \\
  pasoh & 1007 &  &  & \\
  df\_rainfor & 988 &  &  & \\
  nfr\_swi & 729 &  &  & \\
  forst & 537 &  &  & \\
  palanam & 484 &  &  & \\
  unito & 311 &  &  & \\
  uholka & 200 &  &  & \\
  df\_forestplots & 149 &  &  & \\
  mudumalai & 126 &  &  & \\
  lwf\_tree & 114 &  &  & \\
  nwfva\_tree &  84 &  &  & \\
  incds &  75 &  &  & \\
  tuzvo\_tree &  63 &  &  & \\
  iberbas &  57 &  &  & \\
  efm\_swi &  51 &  &  & \\
  france &  47 &  &  & \\
  greece\_stand &  40 &  &  & \\
  czu &  24 &  &  & \\
  ul\_tree &  23 &  &  & \\
  urk &  12 &  &  & \\
  nbw &   7 &  &  & \\
  \bottomrule
  \end{tabular}
\caption{Constituent forest dataset sizes and descriptions.}
\end{table}

\begin{figure}
\centering
\includegraphics[width=0.9\textwidth]{../figures/fig_hist_year.pdf}
\caption{Distribution of forest census data over time, grouped by biome (a-f). Dataset names are explained in Tab. xxxx.}
\end{figure}


\begin{figure}
\centering
\includegraphics[width=0.9\textwidth]{../figures/distribution_length.pdf}
\caption{Distribution of the total length of the time series per forest plot, separated by biomes. The total length corresponds to the difference in the observation year of the first and last available forest inventory for each plot.}
\end{figure}


\begin{figure}
\centering
\includegraphics[width=\textwidth]{../figures/gg_sitepoints.pdf}
\caption{Distribution of forest plots (red circles) and forest biomes.}
\end{figure}

\clearpage

\section{Self-thinning trends}

\begin{figure}
\centering
\includegraphics[width=\textwidth]{../figures/stl_longplots.pdf}
\caption{Self-thinning relation across biomes with example long-term forest monitoring plots highlighted.}
\end{figure}

\begin{figure}
\centering
\includegraphics[width=\textwidth]{../figures/slope_distribution.pdf}
\caption{Self-thinning relation across biomes with example long-term forest monitoring plots highlighted.}
\end{figure}

% XXX todo: visualise evolution of individual forest plots for a small subset of plots, selected based on length of available time series.

% Example for how to include a table, generated externally
% Uses library xtable.
% Function create_table_latex() is included also in this repo.
% R code that was used to create this example table:
% df_experiments_co2_asat <- df5 |>
%   filter(myvar == "asat") |>
%   select(exp, myvar) |>
%   left_join(
%     df |>
%       select(exp, myvar = response, citation),
%     by = join_by(exp, myvar)
%   ) |>
%   select(exp, citation) |>
%   distinct() |>
%
%   # format for processing with latex
%   mutate(citation = paste0('\\', "cite{", citation, "}")) |>
%   group_by(exp) |>
%   summarise(
%     citation = paste0(unique(citation), collapse = ", ")
%   ) |>
%   rename(
%     Experiment = exp,
%     Reference = citation
%   )

% % latex table generated in R 4.5.0 by xtable 1.8-4 package
% Thu Nov 27 09:47:09 2025
\begin{table}[ht]
\centering
\begin{tabular}{lll}
  \hline
Biome & Mean & SD \\ 
  \hline
Boreal Forests/Taiga & 0.30 & 0.06 \\ 
  Mediterranean Forests & 2.35 & 0.06 \\ 
  Temperate Broadleaf \& Mixed Forests & 0.91 & 0.03 \\ 
  Temperate Conifer Forests & 1.18 & 0.06 \\ 
  Tropical \& Subtropical Moist Broadleaf Forests & 0.16 & 0.07 \\ 
  Tropical Dry Broadleaf Forests & -0.38 & 0.46 \\ 
  \end{tabular}
\caption{Percentage change of forest stand density.} 
\end{table}


\begin{table}
\centering
\begin{tabular}{lll}
  \toprule
Biome & Mean & SE \\
  \midrule
  Boreal Forests/Taiga & 0.30 & 0.06 \\
  Mediterranean Forests & 2.35 & 0.06 \\
  Temperate Broadleaf \& Mixed Forests & 0.91 & 0.03 \\
  Temperate Conifer Forests & 1.18 & 0.06 \\
  Tropical \& Subtropical Moist Broadleaf Forests & 0.16 & 0.07 \\
  Tropical Dry Broadleaf Forests & -0.38 & 0.46 \\
  \bottomrule
  \end{tabular}
\caption{Mean estimate and standard error (SE) of percentage change (\%/yr) of forest stand density (number of trees per ha) by biome, determined from quantile regressions on bootstrapped data samples.}
\end{table}


\begin{figure}
\centering
\includegraphics[width=\textwidth]{../figures/hist_percent_change.pdf}
\caption{Distribution of percentage change (\%/yr) in stand density (number of trees per ha) by biome.}
\end{figure}


\begin{figure}
\centering
\includegraphics[width=\textwidth]{../figures/fdisturbed.pdf}
\caption{Trends in the fraction of disturbed forest plots, by biome. Fraction values are logit-transformed. The corresponding un-transformed values are indicated by the right y-axis in each plot. No regression fit is shown for tropical dry broadleaf forests (b) as only two points are available with non-zero values for the disturbed fraction.}
\end{figure}

\clearpage

\section{Environmental drivers}

\begin{table}
\centering
\scriptsize
\begin{talltblr}[         %% tabularray outer open
caption={Regression Results},
]                     %% tabularray outer close
{                     %% tabularray inner open
colspec={Q[]Q[]Q[]Q[]Q[]},
column{2,3,4,5}={}{halign=c,},
column{1}={}{halign=l,},
hline{31}={1,2,3,4,5}{solid, black, 0.05em},
}                     %% tabularray inner close
\toprule
& Complete & No PBR & No PBR, ORGC & No PBR, C:N \\ \midrule %% TinyTableHeader
scale(logQMD) & -0.861*** & -0.862*** & -0.862*** & -0.864*** \\
& [-0.865, -0.856] & [-0.867, -0.858] & [-0.867, -0.857] & [-0.869, -0.859] \\
scale(year) & 0.129*** & 0.130*** & 0.130*** & 0.132*** \\
& [0.126, 0.132] & [0.127, 0.133] & [0.128, 0.133] & [0.129, 0.135] \\
scale(tavg) & -0.033* & -0.026+ & -0.007 & -0.018 \\
& [-0.062, -0.003] & [-0.055, 0.002] & [-0.034, 0.020] & [-0.046, 0.011] \\
scale(ai) & 0.086*** & 0.095*** & 0.097*** & 0.087*** \\
& [0.066, 0.105] & [0.077, 0.114] & [0.079, 0.115] & [0.070, 0.105] \\
scale(ndep) & 0.153*** & 0.140*** & 0.146*** & 0.131*** \\
& [0.133, 0.174] & [0.120, 0.159] & [0.127, 0.166] & [0.112, 0.151] \\
scale(ORGC) & -0.039** & -0.048*** &  & -0.001 \\
& [-0.064, -0.014] & [-0.073, -0.024] &  & [-0.019, 0.017] \\
scale(PBR) & 0.004 &  &  &  \\
& [-0.012, 0.021] &  &  &  \\
scale(CNrt) & 0.057*** & 0.060*** & 0.031*** &  \\
& [0.035, 0.079] & [0.039, 0.081] & [0.015, 0.047] &  \\
scale(year) × scale(tavg) & 0.006** & 0.009*** & 0.013*** & 0.006** \\
& [0.002, 0.011] & [0.005, 0.013] & [0.009, 0.017] & [0.002, 0.010] \\
scale(year) × scale(ai) & -0.022*** & -0.018*** & -0.018*** & -0.017*** \\
& [-0.025, -0.019] & [-0.021, -0.015] & [-0.020, -0.015] & [-0.019, -0.014] \\
scale(year) × scale(ndep) & -0.016*** & -0.015*** & -0.015*** & -0.011*** \\
& [-0.019, -0.013] & [-0.018, -0.012] & [-0.018, -0.012] & [-0.013, -0.008] \\
scale(year) × scale(ORGC) & -0.012*** & -0.011*** &  & -0.028*** \\
& [-0.017, -0.008] & [-0.015, -0.007] &  & [-0.032, -0.025] \\
scale(year) × scale(PBR) & 0.006*** &  &  &  \\
& [0.002, 0.009] &  &  &  \\
scale(year) × scale(CNrt) & -0.021*** & -0.023*** & -0.028*** &  \\
& [-0.025, -0.017] & [-0.026, -0.019] & [-0.031, -0.025] &  \\
SD (Observations) & 0.176 & 0.178 & 0.178 & 0.178 \\
Num.Obs. & 36133 & 37652 & 37652 & 37652 \\
R2 Marg. & 0.521 & 0.530 & 0.531 & 0.527 \\
R2 Cond. & 0.980 & 0.980 & 0.980 & 0.980 \\
AIC & 17693.1 & 19142.8 & 19162.9 & 19315.9 \\
BIC & 17846.0 & 19279.3 & 19282.4 & 19435.4 \\
ICC & 1.0 & 1.0 & 1.0 & 1.0 \\
RMSE & 0.15 & 0.15 & 0.15 & 0.15 \\
\bottomrule
\end{talltblr}
\end{table}


% \begin{table}
\centering
\begin{talltblr}[         %% tabularray outer open
caption={Regression Results},
]                     %% tabularray outer close
{                     %% tabularray inner open
colspec={Q[]Q[]Q[]Q[]Q[]},
column{2,3,4,5}={}{halign=c,},
column{1}={}{halign=l,},
hline{31}={1,2,3,4,5}{solid, black, 0.05em},
}                     %% tabularray inner close
\toprule
& Complete & No PBR & No PBR, ORGC & No PBR, C:N \\ \midrule %% TinyTableHeader
scale(logQMD) & -0.646*** & -0.645*** & -0.645*** & -0.645*** \\
& (0.002) & (0.002) & (0.002) & (0.002) \\
scale(year) & 0.064*** & 0.064*** & 0.065*** & 0.065*** \\
& (0.001) & (0.001) & (0.001) & (0.001) \\
scale(tavg) & 0.025*** & 0.020** & 0.018* & 0.022** \\
& (0.007) & (0.007) & (0.007) & (0.007) \\
scale(ai) & 0.060*** & 0.059*** & 0.059*** & 0.059*** \\
& (0.004) & (0.004) & (0.004) & (0.004) \\
scale(ndep) & 0.042*** & 0.043*** & 0.043*** & 0.042*** \\
& (0.004) & (0.004) & (0.004) & (0.004) \\
scale(ORGC) & 0.007 & 0.005 &  & 0.006 \\
& (0.005) & (0.005) &  & (0.004) \\
scale(PBR) & 0.004 &  &  &  \\
& (0.003) &  &  &  \\
scale(CNrt) & -0.001 & -0.001 & 0.002 &  \\
& (0.004) & (0.004) & (0.003) &  \\
scale(year) × scale(tavg) & -0.007*** & -0.007*** & -0.004* & -0.007*** \\
& (0.002) & (0.001) & (0.001) & (0.001) \\
scale(year) × scale(ai) & -0.009*** & -0.008*** & -0.008*** & -0.007*** \\
& (0.001) & (0.001) & (0.001) & (0.001) \\
scale(year) × scale(ndep) & -0.005*** & -0.006*** & -0.006*** & -0.002 \\
& (0.001) & (0.001) & (0.001) & (0.001) \\
scale(year) × scale(ORGC) & -0.012*** & -0.011*** &  & -0.026*** \\
& (0.002) & (0.002) &  & (0.001) \\
scale(year) × scale(PBR) & 0.003+ &  &  &  \\
& (0.001) &  &  &  \\
scale(year) × scale(CNrt) & -0.019*** & -0.021*** & -0.027*** &  \\
& (0.002) & (0.001) & (0.001) &  \\
SD (Observations) & 0.109 & 0.110 & 0.110 & 0.111 \\
Num.Obs. & 22566 & 23570 & 23570 & 23570 \\
R2 Marg. & 0.728 & 0.712 & 0.710 & 0.711 \\
R2 Cond. & 0.979 & 0.980 & 0.979 & 0.979 \\
AIC & -13634.4 & -13910.0 & -13889.1 & -13732.0 \\
BIC & -13490.0 & -13780.9 & -13776.1 & -13619.0 \\
ICC & 0.9 & 0.9 & 0.9 & 0.9 \\
RMSE & 0.09 & 0.09 & 0.09 & 0.09 \\
\bottomrule
\end{talltblr}
\end{table}


\clearpage

% \begin{figure}
% \centering
% \includegraphics[width=0.8\textwidth]{../figures/coefs_env_quantilefilter.pdf}
% \caption{Coefficients of fixed effects of environmental factors on the self-thinning relationship. Data was filtered to retain observations from plots subject to dominant self-thinning dynamics based on quantiles within QMD bins, following Marqués et al. xxx}
% \end{figure}

\begin{figure}
\centering
\includegraphics[width=0.8\textwidth]{../figures/coefs_env_slopefilter.pdf}
\caption{Coefficients of fixed effects of environmental factors on the self-thinning relationship. Data was filtered to retain observations from plots subject to dominant self-thinning dynamics based on slopes.}
\end{figure}

\clearpage

\section{Global C sink}

\begin{figure}[ht!]
\centering
\includegraphics[width=\textwidth]{../figures/gg_map_sink_perforestarea_combined.pdf}
\caption{(a) C sink in aboveground biomass due to temporal changes in the self-thinning relationship. (b) Standard deviation of estimates across bootstraps. Values are expressed per unit forest area (gC m$^{-2}$ yr$^{-1}$).}
\end{figure}



%%%%%%%%%%%%%%%%%%%%%%%%%%%%%%%%%%%%%%%%%%%%%%%%%%%%%%%%%%%%%%%%%%%%%%%%%%%
\clearpage
\bibliography{references_global_forest_thickening.bib}


\end{document}

